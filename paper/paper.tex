% The MIT License (MIT)
%
% Copyright (c) 2024 Aliaksei Bialiauski
%
% Permission is hereby granted, free of charge, to any person obtaining a copy
% of this software and associated documentation files (the "Software"), to deal
% in the Software without restriction, including without limitation the rights
% to use, copy, modify, merge, publish, distribute, sublicense, and/or sell
% copies of the Software, and to permit persons to whom the Software is
% furnished to do so, subject to the following conditions:
%
% The above copyright notice and this permission notice shall be included
% in all copies or substantial portions of the Software.
%
% THE SOFTWARE IS PROVIDED "AS IS", WITHOUT WARRANTY OF ANY KIND, EXPRESS OR
% IMPLIED, INCLUDING BUT NOT LIMITED TO THE WARRANTIES OF MERCHANTABILITY,
% FITNESS FOR A PARTICULAR PURPOSE AND NON-INFRINGEMENT. IN NO EVENT SHALL THE
% AUTHORS OR COPYRIGHT HOLDERS BE LIABLE FOR ANY CLAIM, DAMAGES OR OTHER
% LIABILITY, WHETHER IN AN ACTION OF CONTRACT, TORT OR OTHERWISE, ARISING FROM,
% OUT OF OR IN CONNECTION WITH THE SOFTWARE OR THE USE OR OTHER DEALINGS IN THE
% SOFTWARE.

\documentclass[sigplan,nonacm,review]{acmart}
\usepackage[utf8]{inputenc}
\usepackage{natbib}
\usepackage{paralist}
\usepackage{ffcode}
\title{?}
\author{Aliaksei Bialiauski}
\orcid{0009-0007-1155-2571}
\email{aliaksei.bialiauski@hey.com}
\affiliation{
    \institution{?}
    \city{Minsk}
    \country{Belarus}
}
\begin{abstract}
This paper is about something new.
\end{abstract}
\keywords{Machine Learning, Text Classification, Random-Forest, Transformers}
\begin{document}
\maketitle


\section{Introduction}\label{sec:introduction}
TBD.. \citet{testCitation}


\section{Related Work}\label{sec:related}


\section{Research Method}\label{sec:method}
The goal of this study is to understand whether GitHub repositories can be
classified as sample or real. For this task we used state-of-art transformer
models for text classification. This leads to the following research questions:
\begin{description}
    \item[RQ1] Can transformers classify GitHub repository on real and
    sample based on text data gained about repository?
    \item[RQ2] Can transformers outperform traditional ML classification
    algorithms like Random-Forest in task of classifying GitHub repositories on
    real and sample?
    \item[RQ3] What is the most suitable technique for classifying GitHub
    repositories on real and sample?
\end{description}

%First, we prepared a training dataset of 1,000 public GitHub repositories.
%It is important to have both: real projects and repositories with examples.
%We distributed number of repositories between real and samples 750 and 250
%respectively. Sample repositories were queried like that: 84 repositories
%that contain \ff{examples} in their name, 83 repositories named with
%\ff{samples}, and 83 contain \ff{guides} in their name.
%For each GitHub repository we collected the following features:
%\begin{inparaenum}[1)]
%    \item \ff{description}: repository's description
%    \item \ff{readme}: README.md file
%    \item \ff{created\_at}: date when repository was created
%    \item \ff{last\_commit}: latest commit date
%    \item \ff{commits}: total amount of commits
%\end{inparaenum}
%
%Second, we label our dataset using numeric labels. The real repository labeled
%as 0, while sample one labeled as 1. To automate the process of labeling we
%utilize pattern-matching script. We run pattern matching for repository's full
%name e.g.: \ff{yegor256/takes}, \ff{apache/kafka}, and it's description. For
%instance, \ff{leeowenowen/rxjava-examples} will match, while
%\ff{objectionary/eo} won't.
%
%% feed different data?
%Third, we prepare dataset to be presented to the model as input. For this
%purpose we preprocess data using techniques of tokenization and stopwords
%removal. After this step, we feed this dataset by learning both, machine
%learning model with Random-Forest algorithm, and deep learning text
%transformers as motivated in RQ2.
%
%Finally, we collect and compare the results, produced by trained models.
%
%All GitHub repositories we collected are public repositories with more than
%20 stars.


\section{Results}\label{sec:results}
TBD..

\section{Limitations}\label{sec:limitations}


\section{Discussion}\label{sec:discussion}


\section{Conclusion}\label{sec:conclusion}


\section{Acknowledgements}\label{sec:acks}

{\raggedright
\bibliographystyle{ACM-Reference-Format}
\bibliography{paper}}
\vfill\eject
\end{document}
