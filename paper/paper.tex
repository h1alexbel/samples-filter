% The MIT License (MIT)
%
% Copyright (c) 2024 Aliaksei Bialiauski
%
% Permission is hereby granted, free of charge, to any person obtaining a copy
% of this software and associated documentation files (the "Software"), to deal
% in the Software without restriction, including without limitation the rights
% to use, copy, modify, merge, publish, distribute, sublicense, and/or sell
% copies of the Software, and to permit persons to whom the Software is
% furnished to do so, subject to the following conditions:
%
% The above copyright notice and this permission notice shall be included
% in all copies or substantial portions of the Software.
%
% THE SOFTWARE IS PROVIDED "AS IS", WITHOUT WARRANTY OF ANY KIND, EXPRESS OR
% IMPLIED, INCLUDING BUT NOT LIMITED TO THE WARRANTIES OF MERCHANTABILITY,
% FITNESS FOR A PARTICULAR PURPOSE AND NON-INFRINGEMENT. IN NO EVENT SHALL THE
% AUTHORS OR COPYRIGHT HOLDERS BE LIABLE FOR ANY CLAIM, DAMAGES OR OTHER
% LIABILITY, WHETHER IN AN ACTION OF CONTRACT, TORT OR OTHERWISE, ARISING FROM,
% OUT OF OR IN CONNECTION WITH THE SOFTWARE OR THE USE OR OTHER DEALINGS IN THE
% SOFTWARE.

\documentclass[sigplan,nonacm,review]{acmart}
\usepackage[utf8]{inputenc}
\usepackage{natbib}
\title{?}
\author{Aliaksei Bialiauski}
\orcid{0009-0007-1155-2571}
\email{aliaksei.bialiauski@hey.com}
\affiliation{
    \institution{?}
    \city{Minsk}
    \country{Belarus}
}
\begin{abstract}
    This paper is about something new.
\end{abstract}
\keywords{Machine Learning, Text Classification, Random-Forest, Transformers}
\begin{document}
\maketitle


\section{Introduction}\label{sec:introduction}
TBD.. \citet{testCitation}


\section{Related Work}\label{sec:related}


\section{Research Method}\label{sec:method}
The goal of this study is to understand whether GitHub repositories can be
classified as sample or real. This leads to the following research questions:
\begin{description}
    \item[RQ1] Do text transformers can predict classes based on text?
    \item[RQ2] Which technique performs better in task to classify GitHub repositories on real and sample?
\end{description}

First, we prepared a dataset of 1,000 public GitHub repositories.
It is important to have both: real projects and repositories with examples.
We distributed number of repositories between real and samples 750 and 250
respectively.

Second, we label our dataset,\ldots


\section{Results}\label{sec:results}
TBD..

\section{Limitations}\label{sec:limitations}


\section{Discussion}\label{sec:discussion}


\section{Conclusion}\label{sec:conclusion}


\section{Acknowledgements}\label{sec:acks}

{\raggedright
\bibliographystyle{ACM-Reference-Format}
\bibliography{paper}}
\vfill\eject
\end{document}
